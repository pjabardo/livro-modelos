\chapter{Um problema simples mas nem tanto}
\label{sec:intro}

Vamos começar com um problema simples: o que cai mais rápido, uma bola de boliche ou uma pena?

A resposta a esta pergunta vai depender de quem respondender. Pergunte a uma criança ou a um adulto sem boa formação científica, a resposta será a bola de boliche. Agora, se a pergunta for feita a um vestibulando ou aluno de graduação das ditas ``ciências exatas'', a resposta será contundente: ambos caem com a mesma velocidade acelerando uniformemente.

Naturalmente, o aluno de ``exatas'' está correto, não é? Qualquer um pode fazer um experimento simples para testar estas hipóteses. Infelizmente o resultado mostrará que a criança tem razão. O aluno de ``exatas'' vai reclamar que isso é injusto pois se não tivermos atrito, ele teria razão. Infelizmente a natureza não liga para estes detalhes. O homem precisou pisar na lua para observar esta solução ``correta'': \url{https://www.youtube.com/watch?v=KDp1tiUsZw8}. Tudo bem, não precisamos necessariamente ir para a lua: uma câmara de vácuo gigante também serve como mostrou Brian Cox em \url{https://www.youtube.com/watch?v=E43-CfukEgs}.

Convenhamos, ir para a lua ou ter acesso a uma câmara de vácuo gigante não é para qualquer um. A dúvida que surge a um observador externo ao sistema educacional científico é: o que é que vocês estão fazendo???

Não, não estamos torturando nossos jovens nem fazendo pegadinhas, estamos introduzindo o método científico. Ao contrário do que se imaginaria ao ler as redes sociais (e grande mídia, infelizmente), a ciência não é um conjunto de fatos que devem ser decorados mas sim uma abordagem sistemática para explorar o mundo. Um dos passos essenciais é simplificar o problema ao máximo. No caso da queda da bola de boliche e da pena, a simplificação consiste em desprezar o atrito. Neste caso o problema fica muito mais simples e as leis fundamentais do universo podem começar a ser exploradas. A importância de Galileu Galilei na ciência não foram os fatos que ele desvendou mas sim a sua abordagem. Um exercício intelectual onde qualquer tipo de atrito é desprezado permitiu-o formular os primeiros princípios da mecânica e o resto, como sabemos,  é história.


Será que não seria melhor sermos mais preciso em nossa análise? O vídeo do Brian Cox também ajuda elucidar esta questão. Antes de soltar a bola de boliche e a pena no vácuo, ele solta na presença do ar. Por um lado, a dinâmica da pena é extremamente complexa. Mas não existe diferença perceptível na dinâmica da bola de boliche. Fazendo uma análise frame a frame da imagem chega-se a uma aceleração de aproximandamente 9.8 $m/s^2$. Para se notar qualquer diferença entre a bola de boliche caindo no vácuo e no ar (pelo menos nos primeiros segundos de queda) seriam necessárias medições cuidadosas, algo muito mais complexo e caro.

Se no caso da bola de boliche, as diferenças são pequenas, a trajetória da pena é extremamente complexa. Observando de perto o momento em que ela é solta, observa-se que há grandes deformações das penas logo no começo (1 min e 39 s no vídeo do Brian Cox acima). Talvez se as penas fossem rígidas a coisa seria mais simples. Solta um pedaço pequeno de papel ou cartolina onde as deformações são pequenas e a trajetória continua sendo complexa.

Tentar analisar a trajetória exata da pena é provavelmente impossível: o problema é certamente caótico e ao se repetir o experimento, o jeitão da trajetória será o mesmo mas cada repetição terá uma trajetória única. Até mesmo caracterizar a geometria e propriedades mecânicas da pena é uma tarefa ingrata. Simplificar o problema, o que aqui chamamos de modelagem é a única saída.

Talvez trabalhando com as penas seja a falta de simetria. Soltando uma placa fina, incialmente ela terá um comportamento próximo do que se observa com a bola de boliche mas não é necessário esperar muito tempo até que a complexidade apareça. Para eliminar quaisquer questões relacionadas a simetria, poderíamos soltar uma esfera bem leve e lisa, talvez uma bexiga, ou bola de criança ou até mesmo ma bola de ping-pong. Mesmo assim, a trajetória pode ser complexa se esperarmos um pouquinho (tente jugar volei com a bexiga...).


Agora é possível entender a genialidade de Galileu. Atacar o problema diretamente mexe com todas as suas nuances  é um problema quase impossível mesmo nos dias de hoje onde todos (ou quase) os fenômenos envolvidos são conhecidos. O problema da queda da pena envolve tudo o que há de mais difícil  em dinâmica, mecânica dos fluidos, elasticidade, biologoia, etc. Este problema ``simples'', é muito mais complexo do que parece e realistacemente sem solução, apenas soluções aproximadas.

Mes este problema da queda de uma pena introduz uma outro problema. Se para Galileu e Newton (e mais tarde Einstein, Heisenberg, Schrödinguer, etc) o problema era determinar os fundamentos da física, hoje em dia, mesmo conhecendo detalhadamente todos (será?) os fenômenos que ocorrem na queda da pena ainda estamos longe de prever de maneira precisa a sua trajetória. O que escreveu Tolstoi em seu livro Anna Karenina, também vale para a física: Famílias felizes são felizes da mesma maneira. Famílias infelizes são miseráveis cad uma de um jeito diferente. A física (principalmente a física do dia a dia) é uma família miserável. Para Galileu, Newton, Einstein, etc a dificuldade era a física ainda não conhecida. Para a maioria mexendo com física aplicada, a dificuldade é a enorme complexidade e a grande quantidade de detalhes presentes em qualquer problema real. Atacar o problema como um todo é impossível. É necessário simplificar o problema.

\section{Modelos}

Mas nem tudo está perdido. A queda da bola de boliche no ar sugere que podemos simplificar um pouco o problema e aplicar a segunda lei de Newton:
\[
m a = m g - F_a
\]
Nesta equação, $m$ é a massa da bola, $a$ é a sua aceleração, $g$ a aceleração da gravidade e $F_a$ é a força de atrito. Se $F_a = m g$, a aceleração é zero e portanto a velocidade é constante. Esta velocidade é conhecida como velocidade terminal. No entanto, se $F_a$ for pequeno, $F_a \ll mg$, o problema é muito semelhante ao problema no vácuo. É isso o que observamos com a queda da bola de boliche no vídeo de Brian Cox.

É lógico que o problema não está resolvido: quanto vale $F_a$? Isso poderia ser obtido realizando experimentos ou calculando a partir das equações da mecânica. Um modelo comum, que será explorado a fundo adiante é:
\[
F_a = C_D A \rho U^2
\]
onde $C_D$ é um coeficiente, $A$ é a área da seção transversal da esfera ($\pi D^2 /4$), $\rho$ é a massa específica do ar (ou qualquer outro fluido) e $U$ é a velocidade de queda. Neste modelo, com o passar do tempo, a velocidade aumenta mas com isso também aumenta a força de arrasto, reduzindo a aceleração. Se esperarmos tempo o suficiente, a força de arrasto irá se igualar com o peso e não haverá mais aceleração. Esta velocidade é conhecida como velocidade terminal. Conhecendo $C_D$, podemos estimar a velocidade terminal:
\[
U_T = \sqrt{\frac{ m g}{C_D A \rho}}
\]

O segredo é não olhar com atenção demais estas equações. $C_D$ varia com a velocidade (mais precisamente com um número adimensional chamado de número de Reynolds e também um outro adimensional, o número de Mach) mas também varia com a orientação do corpo. As forças podem não estar alinhadas na vertical e variar no tempo (estas duas situações ocorrem na realidade), nenhuma bola é uma esfera perfeita  e por aí vai. Mas com cuidado e discernimento, pode-se chegar longe com estes modelos simples. Mas a realidade é muito mais complexa. Quem não se lembra das reclamações sobre a bola Jabulani na copa de 2010. Quem diria que uma bola de futebol poderia causar tantos problemas no século XXI. Nem vou mencionar o quão difícil é jogar bola com uma bexiga. Estes dois exemplos não se resumem à queda vertical de uma esfera mas mostram que a dinâmica de uma esfera em um meio fluido é mais complexo do que parece. O vídeo \url{https://www.youtube.com/watch?v=16Ci_2bN_zc} do excelente canal Veritassium entra em maiores detalhes sobre a velocidade terminal de corpos caindo.

Este modelo ainda tem um problema sério: uma bexiga com hélio ou uma bola de ping-pong dentro d'água vai ter um comportamente bem diferente. Então podemos introduzir um novo termo na equação: o princípio de Arquimedes. No entanto, agora uma nova dificuldade aparece: qual a direção da força de arrasto? Considerando a direção z como vertical para baixo,

\[
m \frac{dU}{dt} = mg -\rho V g - C_D A \rho |U|\cdot U 
\]
onde $z$ é a altura da esfera, e $V$ é o volume da esfera. Agora temos um modelo que funciona, pelo menos funciona mais ou menos se conhecermos algumas constantes. Leva em consideração a força de arrasto, a flutuação e o campo gravitacional. Existe uma caixa preta que é o coeficiente de arrasto $C_D$. Todas as limitações discutidas acima permanecem mas este modelo pode fornecer várias informações interessantes. Para se resolver esta equação diferencial, precisamos das condições iniciais: No instante, $t=0$, $U=U_0$ (no nosso exemplo $U_0 = 0$) e $z=0$.


Mas este problema está longe de ser resolvido. Dependendo da altura em que jogarmos a bola de boliche, teremos variação do campo gravitacional. Ainda existe um outro probla se este experimento for realizado na terra: força de Coriolis. Ao cair em alta velocidade temperaturas altas podem afetar a geometria da bola. Mesmo eliminando todas estas dificuldades relacionadas ao planeta terra (atmosfera e Coriolis) e considerando a gravidade de maneira mais precisa outros fenômenos podem existir: veja a anomalia da trajetória das sondas Pioneer (\url{https://pt.wikipedia.org/wiki/Anomalia_das_Pioneers}).

\section{Escalas}

Admitindo que o corpo parte do repouso e $mg > \rho g V$, então
\[
m \frac{dU}{dt} = (m-\rho V)g - C_D A \rho U^2 
\]

Se estamos falando de uma bola de boliche caindo em ar, o diâmetro da bola de boliche é de aproximadamente 20 cm, volume aproximado de 5 l e área projetada de 350 cm$^2$ e massa de 1 kg. 
\[
\frac{\rho V}{m} \approx 1
\]
Talvez o objetivo seja estimar a velocidade terminal $U_T$. Neste modelo, isto é fácil: $dU/dt = 0$, Talvez seja estimar o tempo que leva para chegar nesta velocidade. 


Se quuisermos levar estimar a velocidade terminal da esfera em alturas onde $g$ varia pouco e $C_D$ é aproximadamente constante, temos a seguinte equação:
\[
mg - \rho g V  = -mg - C_D A \frac{1}{2} \rho |U_T|\cdot U_T
\]

Aguns parâmetros aparencem neste problema, e deste modelo diferentes parâmetros, $m$, $g$, $A$, $\rho$ e  $V$. Estes parâmetro não são independentes. Dado um corpo rígido, existe uma relação entre volume e área. A razão entre os dois é um comprimento que no caso de uma esfera, vale
\[
\frac{V}{A} = \frac{2}{3}\cdot D
\]

Se quisermos saber a velocidade terminal, basta postular que a aceleração é zero:
\[
mg + C_D A \frac{1}{2} \rho |U|\cdot U  = g \rho V
\]
Dividindo esta equação por $g \rho V$ chegamos à seguinte relação:
\[
\left| \frac{m}{\rho V} - 1 \right| = \frac{C_D A U^2}{2 V g}
\]

Lembrando que existe uma relação entre volume e área, e definindo $L = \sqrt[3]{V}$
\[
\left| \frac{m}{\rho L^3} - 1 \right| = k \cdot \frac{U^2}{L g}
\]
onde $k$ é uma constante que depende da geometria e incorpora não apenas a razão entre volume e área mas também o coeficiente de arrasto e constante 2. Este problema, da maneira que está escrito, depende dos seguintes parâmetros:

\begin{itemize}
\item $m$ - massa do corpo
\item $\rho L^3$ - massa do fluido deslocado pelo corpo
\item $Lg$ - Dimensão vezes aceleração da gravidade
\item $k$ - um coeficiente que depende da geometria e aerodinâmica
\end{itemize}

\section{Parâmetros adimensionais}
Nesta última equação, os termos $\frac{m}{\rho L^3}$, $k$ e $\frac{U^2}{L g}$ não possuem unidades, ou seja, mudando o sistema de unidades o valor destas grandezas fica constante. Estas são grandezas adimensionais. Repare que neste modelo, estes parâmetros adimensionais são resultado da razão de forças - dividimos a equação original pela força de flutuação.

\section{Semelhança}
Estes parâmetros adimensionais, produto de manipulação simples do modelo original, tem consequências interessantes. Neste modelo, fixado os parâmetros adimensionais $\frac{m}{\rho L^3}$ e $k$, o outro parâmetro adimensional $\frac{U^2}{Lg}$ é conhecido. Podemos mudar a massa, densidade do fluido, aceleração da gravidade e dimensões do corpo mas enquanto estes parâmetros forem os mesmos, teremos um mesmo valor de $\frac{U^2}{Lg}$. Este é o resultado mais conhecido da Análise Dimensional e é essencial na condução de experimentos em escala reduzida.

\section{Um modelo é uma idealização do problema}
Chegamos a um modelo para a velocidade  problema da queda de um corpo. Para se chegar a este modelo utilizamos uma relação básica da dinâmica, a segunda lei de Newton e um modelo empírico para a força de arrasto de um corpo. A maneira como empregamos a segunda lei de Newton já pressupõe várias simplificações no problema como o fato da força de arrasto estar alinhado com a vertical o que sabemos não ser verdade para a queda de uma pena. Escondemos todas as dificuldades no coeficiente semi-empírico $C_D$. Sabemos que a realidade é muito mais complicada. Mas este modelo pode fornecer informações importantes. Mesmo no caso da queda de uma pena, um coeficiente de arrasto médio pode fornecer informações importantes.

Na pior das hipóteses, um modelo simples permite compreender melhor o problema real e estimar as grandezas importantes do problema, facilitando o projeto de experimentos mais sofisticados.

\section{Estrutura deste artigo}

Com o exemplo do corpo caindo em um meio fluido em um campo gravitacional constante, introduzimos vários conceitos como modelos, parâmetros adimensionais e semelhança. Vamos, agora, entrar mais a fundo em cada questão. Geometria simples será usada para entender melhor os parâmetros adimensionais e semelhança. Em seguida usaremos estes conceitos em alguns problemas simples com equações diferenciais conhecidas, em particular a aerodinâmica de uma esfera. Finalmente apresentaremos a análise dimensional e o teorema dos $\Pi$s de Buckingham.

