\chapter{Exemplos}

\section{Pêndulo simples}

O nosso primeiro exemplo de equação diferencial foi o pêndulo simples. Vamos tentar usar a análise dimensional. O objetivo é tentar determinar o ângulo instantâneo $\theta(t)$ e a reação do pêndulo $N$.

\[
\theta = \theta(t, \theta_0, l, g, m) \qquad N = N(t, \theta_0, l, g, m)
\]

onde $m$ é a massa do pêndulo, $l$ o seu comprimento, $g$ é a aceleração da gravidade, e $\theta_0$ é o ângulo inicial do pêndulo que é adimensional. Vamos usar um sistema de unidades de class $LMT$. Os diferentes parâmetros têm dimensão:

\begin{itemize}
\item $[t] = T$
\item $[\theta_0] = 1$
\item $[l] = L$
\item $[g] = L/T^2$
\item $[m] = M$
\end{itemize}
Já as grandezas que desejamos determinar têm dimensão
\begin{itemize}
\item $[\theta] = 1$
\item $[N] = ML/T^2$
\end{itemize}

Com os parâmetros, temos apenas os seguintes adimensionais:
\begin{itemize}
\item $\Pi_1 = t\sqrt{g/l}$
\item $\Pi_2 = \theta_0$
\end{itemize}

Já as grandezas que queremos determinar apresentam os seguintes adimensionais
\[
\begin{aligned}
  \theta &= \theta\left(\theta_0, t\sqrt{ \frac{g}{l} }\right) \\
  \frac{N}{mg} &= f\left(\theta_0, t\sqrt{ \frac{g}{l} }\right) \\
\end{aligned}
\]

Porque a massa não está do lado direito? Temos 5 parâmetros, sendo que 3 têm dimensões independentes. Um adimensional é o prórpio ângulo inicial $\theta_0$ é o outro é $\Pi_1 = t\sqrt{g/l}$. Pensando fisicamente, isso é natural: a única força externa é o peso que é proporcional à massa mas a aceleração também é proporcional à massa e assim se cancelam. Mas a massa aparece no cálculo da reação $N$. Porque $N$ não foi colocado no lado direito da equação? A reação $N$ não pode ser especificada ou controlada. Ela é resultado da dinâmica do problema, ela é consequência e não causa: existe diferença entre o lado direito e o lado esquerdo da equação.

E o período de oscilação. Neste caso, $t$ pode ser retirado do lado direiro (não é mais um parâmetro) e assom do lado direito só existe o adimensional $\theta_0$. Então temos

\[
T\sqrt{\frac{g}{l}} = f_2(\theta_0)
\]

Será que conseguimos fazer algo a mais com isso? Vejamos o que acontece no limite $\theta_0\longrightarrow 0$. Neste caso existe a solução linear e sabemos que

\[
\lim_{\theta_0\rightarrow 0}T\sqrt{\frac{g}{l}} = f_2(0) = 2\pi
\]

Agora temos uma solução aproximada que vale num limite de um dos adimensionais. Esta solução aproximada poder ser usada como referência:

\[
T = 2\pi\left(\sqrt{\frac{l}{g}}\right) \phi(\theta_0)
\]
onde $\phi(0) = 1$.

\subsection{E se existir amortecimento?}

Precisamos de um modelo para o amortecimento. Poderíamos seguir o que fizemos para a queda de uma bola de boliche mas agora estamos mais preocupados com baixas velocidades. Assim, a força de amortecimento pode ser proporcional à velocidade:

\[
F_a(t) = -c |U|
\]

Agora temos o novo parâmetro $c$ que deve ser considerado. A dimensão de $c$ é
\[
[c] = \frac{M}{T}
\]

Então, 
\[
\theta = \theta(t, \theta_0, l, g, m, c) \qquad N = N(t, \theta_0, l, g, m, c)
\]
Agora temos 6 parâmetros e 3 parâmetros com dimensões independentes:
\[
\begin{aligned}
  \Pi_1 &= t\sqrt{\frac{g}{l}} \\
  \Pi_2 &= \theta_0 \\
  \Pi_3 &= \frac{c}{m}\sqrt{\frac{l}{g}} \\
\end{aligned}
\]

Chamando $\omega_N = \sqrt{g/l}$, a frequência natural, e lembrando da equação do oscilador (massa mola amortecedor),

\[
\Pi_3 = 2\zeta
\]
onde $\zeta$ é a razão de amortecimento. O interessante é que no momento onde introduzimos um amortecimento, a massa volta a ser parte do problema.




